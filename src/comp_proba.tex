\chapter{Compléments de probabilités}

\section{Loi de variables aléatoires réelles}

Rappelons d'abord la définition d'une variable aléatoire réelle.

\begin{definition}
	Soit $(\Omega, \mathcal{A}, \mathcal{P})$ un espace de probabilité.
	Une fonction \GSfunction{$X$}{$\Omega$}{$\real$} est une \textbf{variable
	aléatoire réelle} (v.a.r) si elle est mesurable, c'est-à-dire $\forall A \in
	\mathcal{A}$, $X^{-1}(A) \in \mathcal{B}(\real)$.
\end{definition}

Il faut bien se souvenir qu'une v.a.r n'est qu'une fonction mesurable définie
sur un espace de probabilité!
On définit également, pour chaque v.a.r, sa loi, définie grace à la probabilité
définie sur l'ensemble $\Omega$.

\begin{definition}
	Soit $X$ une v.a.r sur l'espace de probabilité $(\Omega, \mathcal{A},
	\mathcal{P})$. On définit \textbf{la loi de $X$} comme la fonction:

	$P_{X} : (\real, \mathcal{B}(\real)) \rightarrow [0, 1] : A
	\rightarrow P_{X}(A) = P(X^{-1}(A))$.

	On dit que $X$ suit la loi $P_{X}$ et on note $X \sim P_{X}$.
	On peut montrer que $P_{X}$ est une probabilité sur $\real$.
\end{definition}

La loi de $X$ est importante. En effet, celle-ci caractérise la loi pour ce qui
est de son espérance, sa fonction de répartition, sa variance, \ldots.

Une remarque importante est que si on prend 2 v.a.r $X$ et $Y$ sur un même
espace de probabilité $(\Omega, \mathcal{A}, \mathcal{P})$, celle-ci peuvent
prendre des valeurs différentes sur $\Omega$ ($\exists \omega \in \Omega$,
$X(\omega) \neq Y(\omega))$ en ayant pour autant la même loi ($P_{X} = P_{Y}$).

\begin{definition}
	Soit $X$ une v.a.r. On définit \textbf{sa fonction de répartition} $F_{X} :
	\real \rightarrow [0, 1] : x \rightarrow P_{X}(X \leq x)$.
\end{definition}
On distingue \textit{3 types} de v.a.r:

\begin{definition}
	Soit une variable aléatoire $X$. On dit que $X$ est \textbf{une v.a.r
	discrète} si elle prend un nombre fini ou dénombrable de valeurs sur
	$\Omega$.
\end{definition}

\begin{definition}
	Soit une v.a.r $X$. On dit que $X$ est \textbf{une v.a.r continue} si sa loi
	$P_{X}$ est continue.
\end{definition}

\begin{definition}
	Soit une v.a.r $X$. On dit que $X$ est \textbf{une v.a.r mixte} si elle
	n'est ni continue, ni discrète.
\end{definition}

%Exemples

\section{Existence de variable aléatoires réelles}

Nous disons souvent: Soit $X$ une v.a.r suivant la loi $\mathcal{P}$. Mais
comment pouvons nous être sûr que pour tout espace de probabilité, nous pouvons
définir une telle v.a.r? En fait, il est toujours possible, et nous pouvons même
en construire explicitement une.

\begin{proposition}
	\label{var_existence}
	Soit $(\Omega, \mathcal{A}, \mathcal{P})$ un espace de probabilité. Alors il
	existe $X$ v.a.r tel que $X \sim \mathcal{P}$.
\end{proposition}

\begin{proof}
	
\end{proof}

La preuve de \ref{var_existence} peut être étendue pour les vecteur aléatoires:

\begin{proposition}
	Soit $\mathcal{Q}$ une probabilité sur $\real^{d}$. Alors il existe
	$X = (X_{1}, \cdots, X_{d})$ vecteur aléatoire tel que $X \sim \mathcal{Q}$.
\end{proposition}

\begin{proof}
	
\end{proof}

Nous avons alors un corollaire assez puissant se rapportant à l'indépendance des
v.a.r.

\begin{proposition}
	Soit $P_{1}, \cdots, P_{d}$ des probabilités sur $\real$. Alors il existe
	$X_{1}, \cdots, X_{d}$ \textbf{indépendantes} tel que chaque $X_{i} \sim
	P_{i}$.
\end{proposition}

\begin{proof}

\end{proof}

\section{Quantiles}

Prenons le jeu suivant:

Prenons un nombre aléatoire entre 1 et 2000 (par exemple généré avec un
ordinateur), et supposons que votre année de naissance soit paire.
Les règles sont:

\begin{itemize}
	\item Si le résultat est impair, vous perdez 1000€.
	\item Si le résultat est pair, vous gagner 1€.
	\item Si le résultat est votre date de naissance, vous gagnez 2 000 000€.
\end{itemize}
Sous ces règles, allez-vous jouer?

\begin{definition}
	Soit $X$ une v.a.r. On définit \textbf{la médiane de $X$}, et on la note
	$m_{X}$ par $m_{X} = \inf\left\{ x \in \real \, | \, P(X \leq x) \geq
		\frac{1}{2} \right\}$.
\end{definition}

La médiane est la plus petite des valeurs tel que $F_{X}(x) \geq \frac{1}{2}$.
La médiane d'une v.a.r dépend de la loi!

On peut généraliser la définition de la médiane, ce qui nous amène à la
définition de quantile.

\begin{definition}
	Soit $X$ une v.a.r, et $0 < \alpha < 1$. On définit \textbf{le quantile
	d'ordre $\alpha$} ou \textbf{le $\alpha$-quantile}, noté $m_{\alpha}$ comme
	$m_{\alpha} = \inf\left\{ x \in \real \, | \, P(X \leq \alpha) \geq \alpha
	\right\}$.
\end{definition}

La médiane est le $\frac{1}{2}$-quantile.
